\documentclass[10pt,a4paper]{book}
\usepackage[utf8]{inputenc}
\usepackage{amsmath}
\usepackage{amsfonts}
\usepackage{amssymb}
\usepackage{graphicx}
%\usepackage[spanish]{babel}
\usepackage{chemfig}
\usepackage{chemscheme}
\usepackage{chemmacros}
\chemsetup{
	modules=all,
}
\usepackage{siunitx}
\sisetup{%
	output-decimal-marker = {,},
	inter-unit-product = \ensuremath{{}\cdot{}}
}
\usepackage[toc,title,page]{appendix}
\usepackage[hidelinks]{hyperref}
\usepackage{url}
\usepackage{cite}
\usepackage{nomencl}
\usepackage{cancel}
\usepackage{pdfpages}

%----------------------TRADUCCIONES--------------------

\renewcommand*{\listfigurename}{\'Indice de Figuras}
\renewcommand{\listtablename}{\'Indice de Tablas}
\renewcommand{\contentsname}{Contenido}
\renewcommand{\partname}{Parte}
\renewcommand{\figurename}{Figura}
\renewcommand{\tablename}{Tabla}
\renewcommand{\chaptername}{Cap\'itulo} % para ’book’
\renewcommand{\bibname}{Bibliograf\'ia} % para ’book’
\renewcommand{\appendixname}{Anexo}
\renewcommand\appendixtocname{Anexos}
\renewcommand\appendixpagename{ANEXOS}
\renewcommand{\nomname}{\'Indice de S\'imbolos y Abreviaciones}

\usepackage{minted}
\renewcommand{\listingscaption}{C\'odigo}
\renewcommand\listoflistingscaption{\'Indice de c\'odigos fuente}


%\renewcommand{\abstractname}{Resumen} %en article
%\renewcommand{\refname}{Bibliografía} %en article

\newcommand{\rta}{\underline{\textbf{Respuesta:}} }

\begin{document}
%\includepdf[pages=-]{Ideatapa.pdf}

\begin{titlepage}
\begin{center}
{\Huge Soluci\'on: conflictos con {\ttfamily Babel}	}
\end{center}

\end{titlepage}
\tableofcontents % indice de contenidos
\cleardoublepage

%------------------------Construye el indice de Figuras-------------------------------
\addcontentsline{toc}{chapter}{\'{I}ndice de figuras} % para que aparezca en el indice de contenidos
\listoffigures % indice de figuras

%---------------Construye indice de tablas------------------------------------------------
\renewcommand*{\listtablename}{\'{I}ndice de tablas}
\cleardoublepage
\listoftables % indice de tablas
\addcontentsline{toc}{chapter}{\'{I}ndice de tablas} % para que aparezca en el indice de contenidos
\cleardoublepage

%-----------------------Construye el listado de Codigos fuente --------------------------
\listoflistings %To print the list with all listing elements use \listoflistings
\addcontentsline{toc}{chapter}{Lista de c\'odigos fuente} % para que aparezca en el indice de

\chapter{Problemas bloques d y f}
\begin{enumerate}
	
	\item  ¿Se desproporcionan los siguientes iones en condiciones estándar? Calcular \state[subscript-right=r]{G} y la constante de equilibrio para la reacción de dismutación.
	\begin{enumerate}
		\item \ch{Cu\pch{}}
		\item \ch{Hg2\pch[2]}
	\end{enumerate}
	
	\item  Balancear las siguientes ecuaciones de óxido-reducción:
	\begin{enumerate}
		\item \ch{MnO4\mch{} + Co(OH)2\sld{} -> MnO2 + Co(OH)3 \sld{}}
		\item \ch{Ag\sld{} + NO3\mch{} + H3O\pch{} -> Ag\pch{} + NO + H2O}
		\item \ch{Cr(OH)3 + ClO3\mch{} -> CrO4\mch[2] + Cl\mch{}}
		\item \ch{Zn\sld{} + OH\mch{} + H2O -> [Zn(OH)4]\mch[2] + H2}
		\item \ch{AgNO3 + Zn\sld{} -> Zn(NO3)2 + Ag\sld{}}
	\end{enumerate}
	
	\item Para la reacción: \ch{Fe + Zn\pch[2] <=> Fe\pch[2] + Zn}. ¿Cuál es la concentración de equilibrio de \ch{Fe\pch[2]}
	que se alcanza cuando se coloca un trozo de hierro dentro de una disolución \SI{1}{\Molar} de \ch{Zn\pch[2]},
	suponiendo que la concentración de \ch{Zn\pch[2]} no varía?
	
	\item  Una disolución de permanganato de potasio se prepara disolviendo \SI{3,16}{\gram} de la sal en agua y
	llevando el volumen a \SI{1}{\liter}. Calcular:
	\begin{enumerate}
		\item  La molaridad de la solución de permanganato.
		\item El volumen de dicha solución necesario para neutralizar \SI{12}{\milli\liter} de peróxido de hidrógeno
		de \num{5,60} volúmenes.
	\end{enumerate}
	
	\item Una disolución es \SI{0,01}{\Molar} en iones cromato e iones cloruro. Si se le añade a la misma una
	disolución de nitrato de plata:
	\begin{enumerate}
		\item  ¿Qué sal precipita primero?
		\item ¿Cuál es la concentración del ion que precipita en primer lugar cuando se inicia la
		precipitación del segundo?
	\end{enumerate}
	Kps \ch{Ag2CrO4} = \num{1,1e-12}; Kps \ch{AgCl} = \num{1,8e-10}.
	
	\item  ¿Por qué el oro se disuelve en agua regia? Justificar.
	
	\item Explicar por qué se ennegrece la plata al estar expuesta al aire.
	
	\item ¿Qué diferencia al mercurio de los otros elementos del grupo 12?
	\begin{enumerate}
		\item Es un líquido.
		\item Es tóxico.
		\item Forma el catión \ch{Hg2\pch[2]} con un enlace \ch{Hg-Hg}.
	\end{enumerate}
	
	\rta a) Es un líquido; y c) Forma el catión \ch{Hg2\pch[2]} con un enlace \ch{Hg-Hg}.
	
	\item  Se encontró que una muestra de \isotope*{90,Y} aislada recientemente tenía una actividad de \num{9,8e5}
	desintegraciones por minuto a la 1:00 P.M. del 3 de diciembre de 2004. A las 2:15 P.M. del
	17 de diciembre de 2004, su actividad era de \num{2,6e4} desintegraciones por minuto. Calcular la
	vida media del \isotope*{90,Y}.
	
	
	
\end{enumerate}

\appendix
\begin{appendices}
\printnomenclature
	
\chapter{Mi primer anexo}
Ac\'a va el primer anexo

\chapter{Mi segundo anexo}
Ac\'a va el primer anexo
\end{appendices}

\end{document}