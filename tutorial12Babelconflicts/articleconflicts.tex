\documentclass[10pt,a4paper]{article}
\usepackage[utf8]{inputenc}
\usepackage[T1]{fontenc}
\usepackage{amsmath}
\usepackage{amsfonts}
\usepackage{amssymb}
\usepackage{graphicx}
\author{Luis}
\title{Soluci\'on: conflictos con {\ttfamily Babel}}
\date{}
%\usepackage[spanish]{babel}
%\usepackage[spanish,es-noquoting]{babel}
\usepackage[toc,title,page]{appendix}
\usepackage{chemfig}
\usepackage{chemscheme}
\usepackage{chemmacros}
\chemsetup{
	modules=all,
}
\usepackage{siunitx}
\sisetup{%
	output-decimal-marker = {,},
	inter-unit-product = \ensuremath{{}\cdot{}}
}
\usepackage[hidelinks]{hyperref}
\usepackage{url}
\usepackage[sectionbib,square,sort,numbers]{natbib}

\usepackage{nomencl}
\usepackage{cancel}

\newcommand{\rta}{\underline{\textbf{Respuesta:}} }



%----------------------TRADUCCIONES--------------------
\renewcommand{\contentsname}{\'Indice general}
\renewcommand*{\listfigurename}{\'Indice de Figuras}
\renewcommand{\listtablename}{\'Indice de Tablas}
\renewcommand{\figurename}{Figura}
\renewcommand{\tablename}{Tabla}
\renewcommand{\appendixname}{Anexo}
\renewcommand\appendixtocname{Anexos}
\renewcommand\appendixpagename{ANEXOS}
\renewcommand{\refname}{Bibliograf\'ia}
\renewcommand{\nomname}{\'Indice de S\'imbolos y Abreviaciones}
\renewcommand{\abstractname}{Resumen} %en article

%%%%SI USAS BOOK
%\renewcommand{\partname}{Parte}
%\renewcommand\bibname}{Bibliograf\'ia}

%\renewcommand{\listingscaption}{C\'odigo}
%\renewcommand\listoflistingscaption{\'Indice de c\'odigos fuente}



\begin{document}
\maketitle

\tableofcontents % indice de contenidos
%------------------------Construye el indice de Figuras-------------------------------
%\addcontentsline{toc}{section}{\'{I}ndice de figuras} % para que aparezca en el indice de contenidos
\listoffigures % indice de figuras
%\renewcommand*{\listtablename}{\'{I}ndice de tablas}

\listoftables % indice de tablas
\addcontentsline{toc}{section}{\'{I}ndice de tablas}

\section{Problemas bloques d y f}
\begin{enumerate}
	\item Escribir las configuraciones electr\'{o}nicas de: \ch{Sc}, \ch{Ti\pch[4]}, \ch{V\pch[2]}, \ch{Cr}, \ch{Mn\pch[2]}, \ch{Mn\pch[3]}, \ch{Fe\pch[3]}, \ch{Co\pch[3]} , \ch{Zn},
	\ch{Eu\pch[2]}, \ch{U}.
	
	\rta
	\begin{itemize}
		\item \ch{Sc}: \elconf{Sc}
		\item \ch{Ti\pch[4]}: \writeelconf{2,2+6,2+6+2,2}
		\item \ch{V\pch[2]}: \writeelconf{2,2+6,2+6+1,2}
		\item \ch{Cr}: \elconf{Cr}
		\item \ch{Mn\pch[3]}: \writeelconf{2,2+6,2+6+5,1}
		\item \ch{Fe\pch[3]}: \writeelconf{2,2+6,2+6+3,2}
		\item \ch{Co\pch[3]}: \writeelconf{2,2+6,2+6+4,2}
		\item \ch{Zn}: \elconf{Zn}
		\item \ch{Eu\pch[2]}: \writeelconf{2,2+6,2+6+10,2+6+10+5,2+6,2}
		\item \ch{U}: \elconf{U}
	\end{itemize}
	\item Una aleación metálica está constituida por cobre y cinc. Al analizar \SI{0,832}{\gram} de la aleación se
	obtuvieron \SI{0,673}{\gram} de sulfocianuro de cobre (I) y \SI{0,480}{\gram} de pirofosfato de cinc. Calcular el
	porcentaje de cobre y de cinc en dicha aleación.
	
	\rta \ch{Cu}: \SI{63.55}{\gram\per\mole}, \ch{Zn}: \SI{65.38}{\gram\per\mole},\ch{CuSCN}: \SI{21.65}{\gram\per\mole}; \ch{Zn2P2O7}: \SI{304.70}{\gram\per\mole}
	
	\begin{align*}
	\frac{\SI{0.673}{\cancel\gram\of{\ch{CUSCN}}}}{\SI{0.832}{\gram\of{Aleaci\acute{o}n}}} \times \frac{\SI{63.55}{\gram\of{\ch{Cu}}}}{\SI{121.63}{\cancel\gram\of{\ch{CUSCN}}}} \times 100 =& \SI{42.30}{\percent\of{\ch{Cu}}} \\
	\frac{\SI{0.480}{\cancel\gram\of{\ch{Zn2P2O7}}}}{\SI{0.832}{\gram\of{Aleaci\acute{o}n}}} \times \frac{\SI{65.38}{\gram\of{\ch{Zn}}}}{\SI{304.70}{\cancel\gram\of{\ch{ZnP2O7}}}} \times 100 =& \SI{24.70}{\percent\of{\ch{Zn}}}
	\end{align*}
	
	\item Una muestra de \SI{1,63}{\gram} de un óxido de cromo contiene \SI{1,12}{\gram} de cromo. Establecer la fórmula empírica de dicho óxido.
	
	\rta \si{\gram\of{O}} = \SI{1,63}{\gram\of{muestra}} - \SI{1,12}{\gram\of{Cr}} = \SI{0.51}{\gram\of{O}}
	\begin{align*}
	\frac{\SI{1,12}{\gram\of{Cr}}}{\SI{51.99}{\gram\of{Cr}}} = \num{0.021} &\quad \frac{\SI{0.51}{\gram\of{O}}}{\SI{16.0}{\gram\of{O}}} = \num{0.032}\\
	\frac{\num{0.021}}{\num{0.021}} = 1 &\quad \frac{\num{0.032}}{\num{0.021}} \approx \num{1.5}
	\end{align*}
	
	Si multiplico por 2 para obtener n\'umeros enteros, entonces la f\'ormula es: \ch{Cr2O3}
	
	\item  Una de las primeras etapas para la refinación de minerales sulfurados es el procedimiento de
	tostación, en el que el mineral se calienta en presencia de oxígeno para formar un óxido
	metálico y \ch{SO2\gas{}}. Calcular para la tostación de esfalerita (\ch{ZnS}) \state{H} \SI{298}{\kelvin} y \state{G} \SI{298}{\kelvin} de la reacción.
	

		
\end{enumerate}

\appendix
%\addcontentsline{toc}{section}{Anexo}
\renewcommand{\thesection}{\Alph{section}.\arabic{section}}
\setcounter{section}{0}
\begin{appendices}
	\section{Problemas adicionales}
	\begin{enumerate}
		\item  Dada la celda \ch{Zn/Zn\pch[2]//Cu\pch[2]/Cu} a \SI{25}{\degreeCelsius}, calcular el potencial de la misma cuando las concentraciones de \ch{Cu\pch[2]} y \ch{Zn\pch[2]} son iguales.
	
	\item ¿Cuál es el estado de oxidación más común para los lantanoides?
	\begin{enumerate}
		\item +2
		\item +3
		\item +4
	\end{enumerate}
	
	\rta b) +3
	
	\item El mayor estado de oxidación de los actinoides es:
	\begin{enumerate}
		\item +3
		\item +4
		\item +6
	\end{enumerate}
	\end{enumerate}




%\phantomsection
%\section{}
%\printnomenclature

\section{F\'ormulas}
\begin{reactions}
	A &<=> B "\label{reaccio6}" \\
	C &<=> D "\label{reaccionRef5}" \\
	A + C &<=> B + D "\label{reaccion_neta5}"
\end{reactions}

donde la \ref{reaccio6} tiene asociado \gibbs*[subscript-right=desc]{},
\ref{reaccionRef5} \gibbs*[subscript-right=ref]{} y la reacción neta,
\gibbs*[subscript-right=neto]{}. El valor de la energía libre de referencia
puede ser experimental o estimado.
%\newpage
\nocite{*}
\phantomsection
\section{}
%\addcontentsline{toc}{section}{Bibliograf\'ia}
\bibliographystyle{newapa}
\bibliography{biblio}
\end{appendices}

\end{document}