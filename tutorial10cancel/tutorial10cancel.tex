\documentclass[10pt,a4paper]{article}
\usepackage[utf8]{inputenc}
\usepackage{amsmath}
\usepackage{amsfonts}
\usepackage{amssymb}
\usepackage{graphicx}
\usepackage{siunitx}
\sisetup{%
	detect-family,
	%alsoload = synchem,
	%per-mode = fraction,
	%per-mode = symbol,
	%separate-uncertainty = true,
	output-decimal-marker = {,}
	%exponent-product = \cdot
	%inter-unit-product =\cdot
}
\usepackage{chemmacros}
\chemsetup{modules=all}
\usepackage{cancel}
%\usepackage[thicklines, makeroom]{cancel}
\title{Cancel}
\author{Luis}
\date{}

%Definiendo un nuevo estado
\NewChemState\ElPot{ symbol=E , subscript-pos=right , superscript= , unit=\volt}
\NewChemState\ElPotC{ symbol=E ,pre= , subscript-pos=right , subscript-right = C, unit=\volt}
\NewChemState\ElPotA{ symbol=E ,pre= , subscript-pos=right , subscript-right = A, unit=\volt}
\begin{document}
\maketitle

\section{Modos de cancel}
\subsection{cancel}
\begin{equation*}
\frac{x^{\cancel{2}}}{\cancel{x}}
\end{equation*}

\subsection{bcancel}
\begin{equation*}
\frac{j^{\bcancel{2}}}{\bcancel{j}}
\end{equation*}

\subsection{xcancel}
\begin{equation*}
\xcancel{1} + x = \xcancel{1}
\end{equation*}

\subsection{cancelto}
\begin{equation*}
\frac{m^{\cancelto{2}{5}}}{\cancel{m^3}}
\end{equation*}

\section{cancel + siunitx}
\begin{equation*}
\SI{100}{\cancel\milli\liter} \times \frac{\SI{1}{\liter}}{\SI{1000}{\cancel\milli\liter}} = \SI{0.1}{\liter}
\end{equation*}

%\SI{\cancel 12}{\cal}
%$\cancel{\num{e6}}\si{\cal}$

\section{cancel + chemmacros}
\texttt{$\backslash{}$cancel} dentro de \texttt{$\backslash{}$ch}:

\ch{\cancel{Na\pch{}} \aq{}}

\ch{\cancel{2 Na\pch{}} \aq{}}

\noindent Primero \texttt{$\backslash{}$cancel} y adentro \texttt{$\backslash{}$ch}:

$\cancel{\ch{2 Na\pch{}}}$

\section{Ejemplos completos}

\subsection{Factor unitario}
{\it Una muestra de hulla contiene \SI{1,6}{\percent} en masa de azufre. Para evitar la contaminación al quemarla se trata el dióxido de azufre formado con \si{\cal}. Calcular la masa diaria (en \si{\kg}) de óxido de calcio que se necesitan en una planta que utiliza \SI{6,60e6}{\kg} de hulla por día.}

R: \SI{184.800}{\kg}


\begin{equation*}
\SI{6.60e6}{\cancel\kilogram\of{Hulla}}\times\frac{\SI{1,6}{\cancel\kilogram\of{\ch{S}}}}{\SI{100}{\cancel\kilogram\of{Hulla}}}
\times\frac{\SI{0,064}{\cancel\kilogram\of{\ch{SO2}}}}{\SI{0,03206}{\cancel\kilogram\of{\ch{S}}}}
\times\frac{\SI{0,080}{\cancel\kilogram\of{\ch{SO3}}}}{\SI{0,064}{\cancel\kilogram\of{\ch{SO2}}}}
\times\frac{\SI{0,05608}{\kilogram\of{\ch{CaO}}}}{\SI{0,080}{\cancel\kilogram\of{\ch{SO3}}}}
\end{equation*}

\begin{equation*}
= 184717,65\;\si{\kilogram}\;\ch{CaO}
\end{equation*}

{\it Pasar \SI{100}{\milli\liter} a \si{\liter}}



\subsection{Redox}
{\it Escribir las hemirreacciones correspondientes a la carga y descarga de un acumulador de
Pb.} 
\begin{align*}
\ch{PbO2 + HSO4\mch{} + 5 H\pch{} +} \cancel{2 \ch{\el}} &\ch{-> PbSO4 + 2 H2O} &\quad&\ElPotC{-1,70} \\
\ch{Pb\sld{} + SO4\mch[2] &-> PbSO4 +} \cancel{2  \ch{\el}} &\quad&\ElPotA{-0,356} \\
\cline{1-2}
\ch{PbO2 + HSO4\mch{} + 5 H\pch{} + Pb \sld{} &-> 2 PbSO4 + 2 H2O} &\quad& \ElPot{2,06}
\end{align*}
	
\begin{reaction}
{\cancel{2 H2}}
\end{reaction}

\end{document}