\documentclass[10pt,a4paper]{article}
\usepackage[utf8]{inputenc}
\usepackage{amsmath}
\usepackage{amsfonts}
\usepackage{amssymb}
\usepackage{graphicx}
\usepackage{siunitx}
\sisetup{
	%per-mode=symbol
}
\usepackage{chemmacros}
\chemsetup{modules=all}
\title{Chemmacros: Módulo thermodynamics}
\author{Luis}
\date{}

%Definiendo un nuevo estado
\NewChemState\ElPot{ symbol=E , subscript-pos=right , superscript= , unit=\volt}
\NewChemState\ElPotC{ symbol=E ,pre= , subscript-pos=right , subscript-right = C, unit=\volt}
\NewChemState\ElPotA{ symbol=E ,pre= , subscript-pos=right , subscript-right = A, unit=\volt}


\begin{document}
\maketitle
\section{Macro State}
\state{A}, 
\state[superscript-right= ]{A}, 
\state[subscript-left=f]{G}, 
\state[subscript-left=f, superscript-right= ]{G}, 
\state[subscript-right=\ch{Na\pch{}}/\ch{Na}]{E}, 
\state[superscript-right=\SI{1000}{\celsius}]{H}

\subsection{Inclusión en el entorno equations}
La variación de energía de Gibbs para un proceso a temperatura y presión constantes viene dada por: 
\begin{equation*}
\state{G} = \state{H} - T\state{S}
\end{equation*}

El \state[superscript-right=]{G} para una reacción viene dado:
\begin{equation*}
\state[superscript-right=]{G} = \state{G} + RT \ln Q
\end{equation*}

\section{Variables termodinámicas}
El módulo de thermodynamics proporciona algunos comandos para variables termodinámicas específicas.

\enthalpy{123}, \entropy{123}, \gibbs{123}

\subsection{subíndices}
\enthalpy(r){123}, \enthalpy(f){454}

\subsection{Más opciones}
\entropy{56.7}, 
\entropy[pre= ,superscript=]{56.7}
\entropy[pre=$\Delta$,superscript=]{56.7},
\gibbs[post=\ch{CaF2}]{56.7}

\subsection{Modificando las unidades}
\enthalpy[unit=\kilo\joule]{-285.56}, 
\enthalpy[unit=\joule]{-285.56e3}, 
%\enthalpy{-285.56}

\section{Definiendo nuevos estados}
\ElPot{0.22}, \ElPotC{0.44}, \ElPotA{0.55}, 
\\ 
\ElPotC(\ch{Sn\pch[2]}/\ch{Sn}){-0.136}, \ElPotC($\ch{Sn\pch[2]}\mid\ch{Sn}$){-0.136}
\end{document}